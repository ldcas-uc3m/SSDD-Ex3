\section{Compilación y ejecución}

Al igual que en el ejercicio anterior, hemos decidido usar GNU/make para automatizar la compilación.\\
El archivo \texttt{src/Makefile} es similar al anterior, cambiando las dependencias necesarias conforme a la nueva arquitectura y añadiendo las librerías y \textit{flags} del \textit{linker} necesarias para el uso de RPCs.\newline

A la hora de ejecutar, hemos dejado el ejemplo de lanzar un cliente básico (\texttt{src/cliente.c}) en el \textit{script} \texttt{run.sh}, pero es tan sencillo como compilar con GNU/make y definir la IP del servidor mediante la variable de entorno \texttt{IP\_TUPLAS}.\\
Cabe destacar que es necesario que las RPCs estén instaladas en el sistema.

En el makefile también se ha incluido, igual que en el ejercicio anterior, todo lo necesario para compilar los archivos de testeo y que se puedan usar con ambos archivos .sh.